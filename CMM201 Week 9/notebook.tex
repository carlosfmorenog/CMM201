
% Default to the notebook output style

    


% Inherit from the specified cell style.




    
\documentclass[11pt]{article}

    
    
    \usepackage[T1]{fontenc}
    % Nicer default font (+ math font) than Computer Modern for most use cases
    \usepackage{mathpazo}

    % Basic figure setup, for now with no caption control since it's done
    % automatically by Pandoc (which extracts ![](path) syntax from Markdown).
    \usepackage{graphicx}
    % We will generate all images so they have a width \maxwidth. This means
    % that they will get their normal width if they fit onto the page, but
    % are scaled down if they would overflow the margins.
    \makeatletter
    \def\maxwidth{\ifdim\Gin@nat@width>\linewidth\linewidth
    \else\Gin@nat@width\fi}
    \makeatother
    \let\Oldincludegraphics\includegraphics
    % Set max figure width to be 80% of text width, for now hardcoded.
    \renewcommand{\includegraphics}[1]{\Oldincludegraphics[width=.8\maxwidth]{#1}}
    % Ensure that by default, figures have no caption (until we provide a
    % proper Figure object with a Caption API and a way to capture that
    % in the conversion process - todo).
    \usepackage{caption}
    \DeclareCaptionLabelFormat{nolabel}{}
    \captionsetup{labelformat=nolabel}

    \usepackage{adjustbox} % Used to constrain images to a maximum size 
    \usepackage{xcolor} % Allow colors to be defined
    \usepackage{enumerate} % Needed for markdown enumerations to work
    \usepackage{geometry} % Used to adjust the document margins
    \usepackage{amsmath} % Equations
    \usepackage{amssymb} % Equations
    \usepackage{textcomp} % defines textquotesingle
    % Hack from http://tex.stackexchange.com/a/47451/13684:
    \AtBeginDocument{%
        \def\PYZsq{\textquotesingle}% Upright quotes in Pygmentized code
    }
    \usepackage{upquote} % Upright quotes for verbatim code
    \usepackage{eurosym} % defines \euro
    \usepackage[mathletters]{ucs} % Extended unicode (utf-8) support
    \usepackage[utf8x]{inputenc} % Allow utf-8 characters in the tex document
    \usepackage{fancyvrb} % verbatim replacement that allows latex
    \usepackage{grffile} % extends the file name processing of package graphics 
                         % to support a larger range 
    % The hyperref package gives us a pdf with properly built
    % internal navigation ('pdf bookmarks' for the table of contents,
    % internal cross-reference links, web links for URLs, etc.)
    \usepackage{hyperref}
    \usepackage{longtable} % longtable support required by pandoc >1.10
    \usepackage{booktabs}  % table support for pandoc > 1.12.2
    \usepackage[inline]{enumitem} % IRkernel/repr support (it uses the enumerate* environment)
    \usepackage[normalem]{ulem} % ulem is needed to support strikethroughs (\sout)
                                % normalem makes italics be italics, not underlines
    

    
    
    % Colors for the hyperref package
    \definecolor{urlcolor}{rgb}{0,.145,.698}
    \definecolor{linkcolor}{rgb}{.71,0.21,0.01}
    \definecolor{citecolor}{rgb}{.12,.54,.11}

    % ANSI colors
    \definecolor{ansi-black}{HTML}{3E424D}
    \definecolor{ansi-black-intense}{HTML}{282C36}
    \definecolor{ansi-red}{HTML}{E75C58}
    \definecolor{ansi-red-intense}{HTML}{B22B31}
    \definecolor{ansi-green}{HTML}{00A250}
    \definecolor{ansi-green-intense}{HTML}{007427}
    \definecolor{ansi-yellow}{HTML}{DDB62B}
    \definecolor{ansi-yellow-intense}{HTML}{B27D12}
    \definecolor{ansi-blue}{HTML}{208FFB}
    \definecolor{ansi-blue-intense}{HTML}{0065CA}
    \definecolor{ansi-magenta}{HTML}{D160C4}
    \definecolor{ansi-magenta-intense}{HTML}{A03196}
    \definecolor{ansi-cyan}{HTML}{60C6C8}
    \definecolor{ansi-cyan-intense}{HTML}{258F8F}
    \definecolor{ansi-white}{HTML}{C5C1B4}
    \definecolor{ansi-white-intense}{HTML}{A1A6B2}

    % commands and environments needed by pandoc snippets
    % extracted from the output of `pandoc -s`
    \providecommand{\tightlist}{%
      \setlength{\itemsep}{0pt}\setlength{\parskip}{0pt}}
    \DefineVerbatimEnvironment{Highlighting}{Verbatim}{commandchars=\\\{\}}
    % Add ',fontsize=\small' for more characters per line
    \newenvironment{Shaded}{}{}
    \newcommand{\KeywordTok}[1]{\textcolor[rgb]{0.00,0.44,0.13}{\textbf{{#1}}}}
    \newcommand{\DataTypeTok}[1]{\textcolor[rgb]{0.56,0.13,0.00}{{#1}}}
    \newcommand{\DecValTok}[1]{\textcolor[rgb]{0.25,0.63,0.44}{{#1}}}
    \newcommand{\BaseNTok}[1]{\textcolor[rgb]{0.25,0.63,0.44}{{#1}}}
    \newcommand{\FloatTok}[1]{\textcolor[rgb]{0.25,0.63,0.44}{{#1}}}
    \newcommand{\CharTok}[1]{\textcolor[rgb]{0.25,0.44,0.63}{{#1}}}
    \newcommand{\StringTok}[1]{\textcolor[rgb]{0.25,0.44,0.63}{{#1}}}
    \newcommand{\CommentTok}[1]{\textcolor[rgb]{0.38,0.63,0.69}{\textit{{#1}}}}
    \newcommand{\OtherTok}[1]{\textcolor[rgb]{0.00,0.44,0.13}{{#1}}}
    \newcommand{\AlertTok}[1]{\textcolor[rgb]{1.00,0.00,0.00}{\textbf{{#1}}}}
    \newcommand{\FunctionTok}[1]{\textcolor[rgb]{0.02,0.16,0.49}{{#1}}}
    \newcommand{\RegionMarkerTok}[1]{{#1}}
    \newcommand{\ErrorTok}[1]{\textcolor[rgb]{1.00,0.00,0.00}{\textbf{{#1}}}}
    \newcommand{\NormalTok}[1]{{#1}}
    
    % Additional commands for more recent versions of Pandoc
    \newcommand{\ConstantTok}[1]{\textcolor[rgb]{0.53,0.00,0.00}{{#1}}}
    \newcommand{\SpecialCharTok}[1]{\textcolor[rgb]{0.25,0.44,0.63}{{#1}}}
    \newcommand{\VerbatimStringTok}[1]{\textcolor[rgb]{0.25,0.44,0.63}{{#1}}}
    \newcommand{\SpecialStringTok}[1]{\textcolor[rgb]{0.73,0.40,0.53}{{#1}}}
    \newcommand{\ImportTok}[1]{{#1}}
    \newcommand{\DocumentationTok}[1]{\textcolor[rgb]{0.73,0.13,0.13}{\textit{{#1}}}}
    \newcommand{\AnnotationTok}[1]{\textcolor[rgb]{0.38,0.63,0.69}{\textbf{\textit{{#1}}}}}
    \newcommand{\CommentVarTok}[1]{\textcolor[rgb]{0.38,0.63,0.69}{\textbf{\textit{{#1}}}}}
    \newcommand{\VariableTok}[1]{\textcolor[rgb]{0.10,0.09,0.49}{{#1}}}
    \newcommand{\ControlFlowTok}[1]{\textcolor[rgb]{0.00,0.44,0.13}{\textbf{{#1}}}}
    \newcommand{\OperatorTok}[1]{\textcolor[rgb]{0.40,0.40,0.40}{{#1}}}
    \newcommand{\BuiltInTok}[1]{{#1}}
    \newcommand{\ExtensionTok}[1]{{#1}}
    \newcommand{\PreprocessorTok}[1]{\textcolor[rgb]{0.74,0.48,0.00}{{#1}}}
    \newcommand{\AttributeTok}[1]{\textcolor[rgb]{0.49,0.56,0.16}{{#1}}}
    \newcommand{\InformationTok}[1]{\textcolor[rgb]{0.38,0.63,0.69}{\textbf{\textit{{#1}}}}}
    \newcommand{\WarningTok}[1]{\textcolor[rgb]{0.38,0.63,0.69}{\textbf{\textit{{#1}}}}}
    
    
    % Define a nice break command that doesn't care if a line doesn't already
    % exist.
    \def\br{\hspace*{\fill} \\* }
    % Math Jax compatability definitions
    \def\gt{>}
    \def\lt{<}
    % Document parameters
    \title{CMM201\_Week9\_lab\_solved}
    
    
    

    % Pygments definitions
    
\makeatletter
\def\PY@reset{\let\PY@it=\relax \let\PY@bf=\relax%
    \let\PY@ul=\relax \let\PY@tc=\relax%
    \let\PY@bc=\relax \let\PY@ff=\relax}
\def\PY@tok#1{\csname PY@tok@#1\endcsname}
\def\PY@toks#1+{\ifx\relax#1\empty\else%
    \PY@tok{#1}\expandafter\PY@toks\fi}
\def\PY@do#1{\PY@bc{\PY@tc{\PY@ul{%
    \PY@it{\PY@bf{\PY@ff{#1}}}}}}}
\def\PY#1#2{\PY@reset\PY@toks#1+\relax+\PY@do{#2}}

\expandafter\def\csname PY@tok@w\endcsname{\def\PY@tc##1{\textcolor[rgb]{0.73,0.73,0.73}{##1}}}
\expandafter\def\csname PY@tok@c\endcsname{\let\PY@it=\textit\def\PY@tc##1{\textcolor[rgb]{0.25,0.50,0.50}{##1}}}
\expandafter\def\csname PY@tok@cp\endcsname{\def\PY@tc##1{\textcolor[rgb]{0.74,0.48,0.00}{##1}}}
\expandafter\def\csname PY@tok@k\endcsname{\let\PY@bf=\textbf\def\PY@tc##1{\textcolor[rgb]{0.00,0.50,0.00}{##1}}}
\expandafter\def\csname PY@tok@kp\endcsname{\def\PY@tc##1{\textcolor[rgb]{0.00,0.50,0.00}{##1}}}
\expandafter\def\csname PY@tok@kt\endcsname{\def\PY@tc##1{\textcolor[rgb]{0.69,0.00,0.25}{##1}}}
\expandafter\def\csname PY@tok@o\endcsname{\def\PY@tc##1{\textcolor[rgb]{0.40,0.40,0.40}{##1}}}
\expandafter\def\csname PY@tok@ow\endcsname{\let\PY@bf=\textbf\def\PY@tc##1{\textcolor[rgb]{0.67,0.13,1.00}{##1}}}
\expandafter\def\csname PY@tok@nb\endcsname{\def\PY@tc##1{\textcolor[rgb]{0.00,0.50,0.00}{##1}}}
\expandafter\def\csname PY@tok@nf\endcsname{\def\PY@tc##1{\textcolor[rgb]{0.00,0.00,1.00}{##1}}}
\expandafter\def\csname PY@tok@nc\endcsname{\let\PY@bf=\textbf\def\PY@tc##1{\textcolor[rgb]{0.00,0.00,1.00}{##1}}}
\expandafter\def\csname PY@tok@nn\endcsname{\let\PY@bf=\textbf\def\PY@tc##1{\textcolor[rgb]{0.00,0.00,1.00}{##1}}}
\expandafter\def\csname PY@tok@ne\endcsname{\let\PY@bf=\textbf\def\PY@tc##1{\textcolor[rgb]{0.82,0.25,0.23}{##1}}}
\expandafter\def\csname PY@tok@nv\endcsname{\def\PY@tc##1{\textcolor[rgb]{0.10,0.09,0.49}{##1}}}
\expandafter\def\csname PY@tok@no\endcsname{\def\PY@tc##1{\textcolor[rgb]{0.53,0.00,0.00}{##1}}}
\expandafter\def\csname PY@tok@nl\endcsname{\def\PY@tc##1{\textcolor[rgb]{0.63,0.63,0.00}{##1}}}
\expandafter\def\csname PY@tok@ni\endcsname{\let\PY@bf=\textbf\def\PY@tc##1{\textcolor[rgb]{0.60,0.60,0.60}{##1}}}
\expandafter\def\csname PY@tok@na\endcsname{\def\PY@tc##1{\textcolor[rgb]{0.49,0.56,0.16}{##1}}}
\expandafter\def\csname PY@tok@nt\endcsname{\let\PY@bf=\textbf\def\PY@tc##1{\textcolor[rgb]{0.00,0.50,0.00}{##1}}}
\expandafter\def\csname PY@tok@nd\endcsname{\def\PY@tc##1{\textcolor[rgb]{0.67,0.13,1.00}{##1}}}
\expandafter\def\csname PY@tok@s\endcsname{\def\PY@tc##1{\textcolor[rgb]{0.73,0.13,0.13}{##1}}}
\expandafter\def\csname PY@tok@sd\endcsname{\let\PY@it=\textit\def\PY@tc##1{\textcolor[rgb]{0.73,0.13,0.13}{##1}}}
\expandafter\def\csname PY@tok@si\endcsname{\let\PY@bf=\textbf\def\PY@tc##1{\textcolor[rgb]{0.73,0.40,0.53}{##1}}}
\expandafter\def\csname PY@tok@se\endcsname{\let\PY@bf=\textbf\def\PY@tc##1{\textcolor[rgb]{0.73,0.40,0.13}{##1}}}
\expandafter\def\csname PY@tok@sr\endcsname{\def\PY@tc##1{\textcolor[rgb]{0.73,0.40,0.53}{##1}}}
\expandafter\def\csname PY@tok@ss\endcsname{\def\PY@tc##1{\textcolor[rgb]{0.10,0.09,0.49}{##1}}}
\expandafter\def\csname PY@tok@sx\endcsname{\def\PY@tc##1{\textcolor[rgb]{0.00,0.50,0.00}{##1}}}
\expandafter\def\csname PY@tok@m\endcsname{\def\PY@tc##1{\textcolor[rgb]{0.40,0.40,0.40}{##1}}}
\expandafter\def\csname PY@tok@gh\endcsname{\let\PY@bf=\textbf\def\PY@tc##1{\textcolor[rgb]{0.00,0.00,0.50}{##1}}}
\expandafter\def\csname PY@tok@gu\endcsname{\let\PY@bf=\textbf\def\PY@tc##1{\textcolor[rgb]{0.50,0.00,0.50}{##1}}}
\expandafter\def\csname PY@tok@gd\endcsname{\def\PY@tc##1{\textcolor[rgb]{0.63,0.00,0.00}{##1}}}
\expandafter\def\csname PY@tok@gi\endcsname{\def\PY@tc##1{\textcolor[rgb]{0.00,0.63,0.00}{##1}}}
\expandafter\def\csname PY@tok@gr\endcsname{\def\PY@tc##1{\textcolor[rgb]{1.00,0.00,0.00}{##1}}}
\expandafter\def\csname PY@tok@ge\endcsname{\let\PY@it=\textit}
\expandafter\def\csname PY@tok@gs\endcsname{\let\PY@bf=\textbf}
\expandafter\def\csname PY@tok@gp\endcsname{\let\PY@bf=\textbf\def\PY@tc##1{\textcolor[rgb]{0.00,0.00,0.50}{##1}}}
\expandafter\def\csname PY@tok@go\endcsname{\def\PY@tc##1{\textcolor[rgb]{0.53,0.53,0.53}{##1}}}
\expandafter\def\csname PY@tok@gt\endcsname{\def\PY@tc##1{\textcolor[rgb]{0.00,0.27,0.87}{##1}}}
\expandafter\def\csname PY@tok@err\endcsname{\def\PY@bc##1{\setlength{\fboxsep}{0pt}\fcolorbox[rgb]{1.00,0.00,0.00}{1,1,1}{\strut ##1}}}
\expandafter\def\csname PY@tok@kc\endcsname{\let\PY@bf=\textbf\def\PY@tc##1{\textcolor[rgb]{0.00,0.50,0.00}{##1}}}
\expandafter\def\csname PY@tok@kd\endcsname{\let\PY@bf=\textbf\def\PY@tc##1{\textcolor[rgb]{0.00,0.50,0.00}{##1}}}
\expandafter\def\csname PY@tok@kn\endcsname{\let\PY@bf=\textbf\def\PY@tc##1{\textcolor[rgb]{0.00,0.50,0.00}{##1}}}
\expandafter\def\csname PY@tok@kr\endcsname{\let\PY@bf=\textbf\def\PY@tc##1{\textcolor[rgb]{0.00,0.50,0.00}{##1}}}
\expandafter\def\csname PY@tok@bp\endcsname{\def\PY@tc##1{\textcolor[rgb]{0.00,0.50,0.00}{##1}}}
\expandafter\def\csname PY@tok@fm\endcsname{\def\PY@tc##1{\textcolor[rgb]{0.00,0.00,1.00}{##1}}}
\expandafter\def\csname PY@tok@vc\endcsname{\def\PY@tc##1{\textcolor[rgb]{0.10,0.09,0.49}{##1}}}
\expandafter\def\csname PY@tok@vg\endcsname{\def\PY@tc##1{\textcolor[rgb]{0.10,0.09,0.49}{##1}}}
\expandafter\def\csname PY@tok@vi\endcsname{\def\PY@tc##1{\textcolor[rgb]{0.10,0.09,0.49}{##1}}}
\expandafter\def\csname PY@tok@vm\endcsname{\def\PY@tc##1{\textcolor[rgb]{0.10,0.09,0.49}{##1}}}
\expandafter\def\csname PY@tok@sa\endcsname{\def\PY@tc##1{\textcolor[rgb]{0.73,0.13,0.13}{##1}}}
\expandafter\def\csname PY@tok@sb\endcsname{\def\PY@tc##1{\textcolor[rgb]{0.73,0.13,0.13}{##1}}}
\expandafter\def\csname PY@tok@sc\endcsname{\def\PY@tc##1{\textcolor[rgb]{0.73,0.13,0.13}{##1}}}
\expandafter\def\csname PY@tok@dl\endcsname{\def\PY@tc##1{\textcolor[rgb]{0.73,0.13,0.13}{##1}}}
\expandafter\def\csname PY@tok@s2\endcsname{\def\PY@tc##1{\textcolor[rgb]{0.73,0.13,0.13}{##1}}}
\expandafter\def\csname PY@tok@sh\endcsname{\def\PY@tc##1{\textcolor[rgb]{0.73,0.13,0.13}{##1}}}
\expandafter\def\csname PY@tok@s1\endcsname{\def\PY@tc##1{\textcolor[rgb]{0.73,0.13,0.13}{##1}}}
\expandafter\def\csname PY@tok@mb\endcsname{\def\PY@tc##1{\textcolor[rgb]{0.40,0.40,0.40}{##1}}}
\expandafter\def\csname PY@tok@mf\endcsname{\def\PY@tc##1{\textcolor[rgb]{0.40,0.40,0.40}{##1}}}
\expandafter\def\csname PY@tok@mh\endcsname{\def\PY@tc##1{\textcolor[rgb]{0.40,0.40,0.40}{##1}}}
\expandafter\def\csname PY@tok@mi\endcsname{\def\PY@tc##1{\textcolor[rgb]{0.40,0.40,0.40}{##1}}}
\expandafter\def\csname PY@tok@il\endcsname{\def\PY@tc##1{\textcolor[rgb]{0.40,0.40,0.40}{##1}}}
\expandafter\def\csname PY@tok@mo\endcsname{\def\PY@tc##1{\textcolor[rgb]{0.40,0.40,0.40}{##1}}}
\expandafter\def\csname PY@tok@ch\endcsname{\let\PY@it=\textit\def\PY@tc##1{\textcolor[rgb]{0.25,0.50,0.50}{##1}}}
\expandafter\def\csname PY@tok@cm\endcsname{\let\PY@it=\textit\def\PY@tc##1{\textcolor[rgb]{0.25,0.50,0.50}{##1}}}
\expandafter\def\csname PY@tok@cpf\endcsname{\let\PY@it=\textit\def\PY@tc##1{\textcolor[rgb]{0.25,0.50,0.50}{##1}}}
\expandafter\def\csname PY@tok@c1\endcsname{\let\PY@it=\textit\def\PY@tc##1{\textcolor[rgb]{0.25,0.50,0.50}{##1}}}
\expandafter\def\csname PY@tok@cs\endcsname{\let\PY@it=\textit\def\PY@tc##1{\textcolor[rgb]{0.25,0.50,0.50}{##1}}}

\def\PYZbs{\char`\\}
\def\PYZus{\char`\_}
\def\PYZob{\char`\{}
\def\PYZcb{\char`\}}
\def\PYZca{\char`\^}
\def\PYZam{\char`\&}
\def\PYZlt{\char`\<}
\def\PYZgt{\char`\>}
\def\PYZsh{\char`\#}
\def\PYZpc{\char`\%}
\def\PYZdl{\char`\$}
\def\PYZhy{\char`\-}
\def\PYZsq{\char`\'}
\def\PYZdq{\char`\"}
\def\PYZti{\char`\~}
% for compatibility with earlier versions
\def\PYZat{@}
\def\PYZlb{[}
\def\PYZrb{]}
\makeatother


    % Exact colors from NB
    \definecolor{incolor}{rgb}{0.0, 0.0, 0.5}
    \definecolor{outcolor}{rgb}{0.545, 0.0, 0.0}



    
    % Prevent overflowing lines due to hard-to-break entities
    \sloppy 
    % Setup hyperref package
    \hypersetup{
      breaklinks=true,  % so long urls are correctly broken across lines
      colorlinks=true,
      urlcolor=urlcolor,
      linkcolor=linkcolor,
      citecolor=citecolor,
      }
    % Slightly bigger margins than the latex defaults
    
    \geometry{verbose,tmargin=1in,bmargin=1in,lmargin=1in,rmargin=1in}
    
    

    \begin{document}
    
    
    \maketitle
    
    

    
    First, download image ``logo.png'' from Moodle, which correspond to a
grayscale version of the Python logo. Save these file \textbf{in the
same directory as this Jupyter notebook}.

    To work with images in Python, we will use packages such as
\textbf{numpy}, \textbf{PIL}, \textbf{skimage} and \textbf{matplotlib}.
Execute the code in the following cell to install these packages using
the \emph{pip} command (remember that we add the ! since this command is
used in the terminal):

    \begin{Verbatim}[commandchars=\\\{\}]
{\color{incolor}In [{\color{incolor} }]:} \PY{o}{!}pip install pillow
        \PY{o}{!}pip install matplotlib
        \PY{o}{!}pip install scikit\PYZhy{}image
\end{Verbatim}


    Notice that we did not install numpy due to the following reasons: * You
have installed it before during class (and there is no need to install
it again). * Sometimes packages include sub-packages. Such is the case
of the three previous ones, which already contain numpy.

    \hypertarget{importing-the-modules}{%
\subsection{Importing the modules}\label{importing-the-modules}}

    Now we will import the packages, functions and classes we will use for
this activity:

    \begin{itemize}
\tightlist
\item
  The first one is the \textbf{Image} class from the \textbf{PIL} module
  (which comes in the \textbf{pillow} package). For this, we will
  execute the following command:
\end{itemize}

    \begin{Verbatim}[commandchars=\\\{\}]
{\color{incolor}In [{\color{incolor}1}]:} \PY{k+kn}{from} \PY{n+nn}{PIL} \PY{k}{import} \PY{n}{Image}
\end{Verbatim}


    \begin{itemize}
\tightlist
\item
  To handle and perform operations on the images, we need
  \textbf{numpy}, which can be imported by executing the command we saw
  in class:
\end{itemize}

    \begin{Verbatim}[commandchars=\\\{\}]
{\color{incolor}In [{\color{incolor}2}]:} \PY{k+kn}{import} \PY{n+nn}{numpy} \PY{k}{as} \PY{n+nn}{np}
\end{Verbatim}


    \begin{itemize}
\tightlist
\item
  To plot the images, we will use \textbf{matplotlib.pyplot} through the
  pseudonym \textbf{plt}:
\end{itemize}

    \begin{Verbatim}[commandchars=\\\{\}]
{\color{incolor}In [{\color{incolor}3}]:} \PY{k+kn}{import} \PY{n+nn}{matplotlib}\PY{n+nn}{.}\PY{n+nn}{pyplot} \PY{k}{as} \PY{n+nn}{plt}
\end{Verbatim}


    \hypertarget{working-with-grayscale-images}{%
\subsection{Working with Grayscale
Images}\label{working-with-grayscale-images}}

    Now it's time to import the grayscale logo to Python. To do so, we can
use the \textbf{open} function from the \textbf{Image} class, storing
the image in a variable called \textbf{img}:

    \begin{Verbatim}[commandchars=\\\{\}]
{\color{incolor}In [{\color{incolor}4}]:} \PY{n}{img} \PY{o}{=} \PY{n}{Image}\PY{o}{.}\PY{n}{open}\PY{p}{(}\PY{l+s+s2}{\PYZdq{}}\PY{l+s+s2}{logo.png}\PY{l+s+s2}{\PYZdq{}}\PY{p}{)}
\end{Verbatim}


    If we check the type of the image, we will see that it belongs to a
weird \textbf{PIL.Image\ldots{}} type.

    \begin{Verbatim}[commandchars=\\\{\}]
{\color{incolor}In [{\color{incolor}5}]:} \PY{n+nb}{type}\PY{p}{(}\PY{n}{img}\PY{p}{)}
\end{Verbatim}


\begin{Verbatim}[commandchars=\\\{\}]
{\color{outcolor}Out[{\color{outcolor}5}]:} PIL.PngImagePlugin.PngImageFile
\end{Verbatim}
            
    To manipulate the image, we need to convert it into a numpy array. To do
so, we can use the following command:

    \begin{Verbatim}[commandchars=\\\{\}]
{\color{incolor}In [{\color{incolor}6}]:} \PY{n}{img} \PY{o}{=} \PY{n}{np}\PY{o}{.}\PY{n}{array}\PY{p}{(}\PY{n}{img}\PY{p}{)}
\end{Verbatim}


    Now check the type of img in the following cell:

    \begin{Verbatim}[commandchars=\\\{\}]
{\color{incolor}In [{\color{incolor}7}]:} \PY{n+nb}{type}\PY{p}{(}\PY{n}{img}\PY{p}{)}
\end{Verbatim}


\begin{Verbatim}[commandchars=\\\{\}]
{\color{outcolor}Out[{\color{outcolor}7}]:} numpy.ndarray
\end{Verbatim}
            
    One particularity of the Image.open() function is that, by default, it
imports all images as colour ones! This means that although the file
contains a grayscale image, the numpy array that we have imported has
three dimensions (keep in mind that images within the computer are
typically composed of three channels: red, green and blue). An example
of this is shown by running the following code (don't get too much into
it, just run it!):

    \begin{Verbatim}[commandchars=\\\{\}]
{\color{incolor}In [{\color{incolor}8}]:} \PY{k+kn}{from} \PY{n+nn}{skimage} \PY{k}{import} \PY{n}{data}
        \PY{n}{cat} \PY{o}{=} \PY{n}{np}\PY{o}{.}\PY{n}{asarray}\PY{p}{(}\PY{n}{data}\PY{o}{.}\PY{n}{chelsea}\PY{p}{(}\PY{p}{)}\PY{p}{)}
        \PY{n+nb}{print}\PY{p}{(}\PY{n}{cat}\PY{o}{.}\PY{n}{shape}\PY{p}{)}
        \PY{n}{cat\PYZus{}R} \PY{o}{=} \PY{n}{cat}\PY{o}{.}\PY{n}{copy}\PY{p}{(}\PY{p}{)}
        \PY{n}{cat\PYZus{}R}\PY{p}{[}\PY{p}{:}\PY{p}{,} \PY{p}{:}\PY{p}{,} \PY{p}{(}\PY{l+m+mi}{1}\PY{p}{,} \PY{l+m+mi}{2}\PY{p}{)}\PY{p}{]} \PY{o}{=} \PY{l+m+mi}{0}
        \PY{n}{cat\PYZus{}G} \PY{o}{=} \PY{n}{cat}\PY{o}{.}\PY{n}{copy}\PY{p}{(}\PY{p}{)}
        \PY{n}{cat\PYZus{}G}\PY{p}{[}\PY{p}{:}\PY{p}{,} \PY{p}{:}\PY{p}{,} \PY{p}{(}\PY{l+m+mi}{0}\PY{p}{,} \PY{l+m+mi}{2}\PY{p}{)}\PY{p}{]} \PY{o}{=} \PY{l+m+mi}{0}
        \PY{n}{cat\PYZus{}B} \PY{o}{=} \PY{n}{cat}\PY{o}{.}\PY{n}{copy}\PY{p}{(}\PY{p}{)}
        \PY{n}{cat\PYZus{}B}\PY{p}{[}\PY{p}{:}\PY{p}{,} \PY{p}{:}\PY{p}{,} \PY{p}{(}\PY{l+m+mi}{0}\PY{p}{,} \PY{l+m+mi}{1}\PY{p}{)}\PY{p}{]} \PY{o}{=} \PY{l+m+mi}{0}
        \PY{n}{cat\PYZus{}RGB} \PY{o}{=} \PY{n}{np}\PY{o}{.}\PY{n}{concatenate}\PY{p}{(}\PY{p}{(}\PY{n}{cat}\PY{p}{,} \PY{n}{cat\PYZus{}R}\PY{p}{,} \PY{n}{cat\PYZus{}G}\PY{p}{,} \PY{n}{cat\PYZus{}B}\PY{p}{)}\PY{p}{,} \PY{n}{axis}\PY{o}{=}\PY{l+m+mi}{1}\PY{p}{)}
        \PY{n}{plt}\PY{o}{.}\PY{n}{imshow}\PY{p}{(}\PY{n}{cat\PYZus{}RGB}\PY{p}{)}
\end{Verbatim}


    \begin{Verbatim}[commandchars=\\\{\}]
(300, 451, 3)

    \end{Verbatim}

\begin{Verbatim}[commandchars=\\\{\}]
{\color{outcolor}Out[{\color{outcolor}8}]:} <matplotlib.image.AxesImage at 0xdc48350>
\end{Verbatim}
            
    \begin{center}
    \adjustimage{max size={0.9\linewidth}{0.9\paperheight}}{output_22_2.png}
    \end{center}
    { \hspace*{\fill} \\}
    
    Therefore, if we check the dimensions of \textbf{img}, we will see that
it has three (one per channel):

    \begin{Verbatim}[commandchars=\\\{\}]
{\color{incolor}In [{\color{incolor}9}]:} \PY{n+nb}{print}\PY{p}{(}\PY{n}{img}\PY{o}{.}\PY{n}{ndim}\PY{p}{)}
\end{Verbatim}


    \begin{Verbatim}[commandchars=\\\{\}]
3

    \end{Verbatim}

    At this stage we don't want this, therefore we will use another module
to import the image as a \emph{grayscale image}. Therefore, we will use
the \textbf{color} and \textbf{io} classes from \textbf{scikit-image}
(referred to as \textbf{skimage}) to import the grayscale image into the
\textbf{img} variable using the following code:

    \begin{Verbatim}[commandchars=\\\{\}]
{\color{incolor}In [{\color{incolor}10}]:} \PY{k+kn}{from} \PY{n+nn}{skimage} \PY{k}{import} \PY{n}{color}
         \PY{k+kn}{from} \PY{n+nn}{skimage} \PY{k}{import} \PY{n}{io}
         
         \PY{n}{img} \PY{o}{=} \PY{n}{color}\PY{o}{.}\PY{n}{rgb2gray}\PY{p}{(}\PY{n}{io}\PY{o}{.}\PY{n}{imread}\PY{p}{(}\PY{l+s+s1}{\PYZsq{}}\PY{l+s+s1}{logo.png}\PY{l+s+s1}{\PYZsq{}}\PY{p}{)}\PY{p}{)}
\end{Verbatim}


    Now check the type and the dimensions of \textbf{img} to verify that now
it stores a two-dimensional numpy array.

    \begin{Verbatim}[commandchars=\\\{\}]
{\color{incolor}In [{\color{incolor}11}]:} \PY{c+c1}{\PYZsh{}\PYZsh{} Use this cell to print the type and the dimension of img}
         \PY{n+nb}{print}\PY{p}{(}\PY{n+nb}{type}\PY{p}{(}\PY{n}{img}\PY{p}{)}\PY{p}{,} \PY{n}{img}\PY{o}{.}\PY{n}{ndim}\PY{p}{)}
\end{Verbatim}


    \begin{Verbatim}[commandchars=\\\{\}]
<class 'numpy.ndarray'> 2

    \end{Verbatim}

    You can also check the number of pixels that the image contains. Notice
that the original file is a 75x75 pixel sized image, and therefore your
\textbf{img} numpy array should have that same size.

    \begin{Verbatim}[commandchars=\\\{\}]
{\color{incolor}In [{\color{incolor}12}]:} \PY{n}{img}\PY{o}{.}\PY{n}{shape}
\end{Verbatim}


\begin{Verbatim}[commandchars=\\\{\}]
{\color{outcolor}Out[{\color{outcolor}12}]:} (75, 75)
\end{Verbatim}
            
    What would happen if we print \textbf{img}? Are we able to see the
image? Run the following cell to find out!

    \begin{Verbatim}[commandchars=\\\{\}]
{\color{incolor}In [{\color{incolor}13}]:} \PY{n+nb}{print}\PY{p}{(}\PY{n}{img}\PY{p}{)}
\end{Verbatim}


    \begin{Verbatim}[commandchars=\\\{\}]
[[1. 1. 1. {\ldots} 1. 1. 1.]
 [1. 1. 1. {\ldots} 1. 1. 1.]
 [1. 1. 1. {\ldots} 1. 1. 1.]
 {\ldots}
 [1. 1. 1. {\ldots} 1. 1. 1.]
 [1. 1. 1. {\ldots} 1. 1. 1.]
 [1. 1. 1. {\ldots} 1. 1. 1.]]

    \end{Verbatim}

    Notice that we only see a bunch of ``ones'' in lists. This is because
for now, the image is a numpy array with values from 0 to 1, where 0
represents black and 1 represents white. Since the print command only
shows us the ``corners'' of the numpy array, we are only able to see the
white corners of the images (hence the ones).

    To show the actual image, we can use the \textbf{plt.imshow} function
from the \textbf{matplotlib.pyplot} library as follows:

    \begin{Verbatim}[commandchars=\\\{\}]
{\color{incolor}In [{\color{incolor}14}]:} \PY{n}{plt}\PY{o}{.}\PY{n}{imshow}\PY{p}{(}\PY{n}{img}\PY{p}{)}
\end{Verbatim}


\begin{Verbatim}[commandchars=\\\{\}]
{\color{outcolor}Out[{\color{outcolor}14}]:} <matplotlib.image.AxesImage at 0x1483630>
\end{Verbatim}
            
    \begin{center}
    \adjustimage{max size={0.9\linewidth}{0.9\paperheight}}{output_35_1.png}
    \end{center}
    { \hspace*{\fill} \\}
    
    It is likely that the image has a yellow background. This will help us
to better identify the background pixels.

    \hypertarget{locating-the-value-of-pixels}{%
\subsubsection{Locating the value of
pixels}\label{locating-the-value-of-pixels}}

    Notice that the image is shown with a coordinate plane which gives us an
idea of the positions of the pixels within the logo. Given that this is
a 75x75 image, we can access different positions of the image to see
their value. For instance, to see the value of the left-top pixel we can
use the following command:

    \begin{Verbatim}[commandchars=\\\{\}]
{\color{incolor}In [{\color{incolor}15}]:} \PY{n+nb}{print}\PY{p}{(}\PY{n}{img}\PY{p}{[}\PY{l+m+mi}{0}\PY{p}{,}\PY{l+m+mi}{0}\PY{p}{]}\PY{p}{)}
\end{Verbatim}


    \begin{Verbatim}[commandchars=\\\{\}]
1.0

    \end{Verbatim}

    The first zero corresponds to the y-axis (rows) and the second zero
corresponds to the x-axis (columns). We can use the same command to see
different positions of the image. For instance, you can see the value of
the other corners or some values from the two pythons.

    \begin{Verbatim}[commandchars=\\\{\}]
{\color{incolor}In [{\color{incolor}16}]:} \PY{c+c1}{\PYZsh{}\PYZsh{} Use this cell to print the values of different pixels}
         \PY{n+nb}{print}\PY{p}{(}\PY{n}{img}\PY{p}{[}\PY{l+m+mi}{20}\PY{p}{,}\PY{l+m+mi}{40}\PY{p}{]}\PY{p}{)}
\end{Verbatim}


    \begin{Verbatim}[commandchars=\\\{\}]
0.196078431372549

    \end{Verbatim}

    Write down the numerical values corresponding to each of the pythons'
bodies. We will use this information later!

    \hypertarget{changing-the-value-of-pixels}{%
\subsubsection{Changing the value of
pixels}\label{changing-the-value-of-pixels}}

    It is also possible to change the value of a pixel. To do so, you can
assign a new value to a certain location. For instance, we can paint the
top-left pixel (coordinate 0,0) in black with the following code:

    \begin{Verbatim}[commandchars=\\\{\}]
{\color{incolor}In [{\color{incolor}17}]:} \PY{n}{img}\PY{p}{[}\PY{l+m+mi}{0}\PY{p}{,}\PY{l+m+mi}{0}\PY{p}{]}\PY{o}{=}\PY{l+m+mi}{0}
         \PY{n+nb}{print}\PY{p}{(}\PY{n}{img}\PY{p}{)}
         \PY{n}{plt}\PY{o}{.}\PY{n}{imshow}\PY{p}{(}\PY{n}{img}\PY{p}{)}
\end{Verbatim}


    \begin{Verbatim}[commandchars=\\\{\}]
[[0. 1. 1. {\ldots} 1. 1. 1.]
 [1. 1. 1. {\ldots} 1. 1. 1.]
 [1. 1. 1. {\ldots} 1. 1. 1.]
 {\ldots}
 [1. 1. 1. {\ldots} 1. 1. 1.]
 [1. 1. 1. {\ldots} 1. 1. 1.]
 [1. 1. 1. {\ldots} 1. 1. 1.]]

    \end{Verbatim}

\begin{Verbatim}[commandchars=\\\{\}]
{\color{outcolor}Out[{\color{outcolor}17}]:} <matplotlib.image.AxesImage at 0x14c0e10>
\end{Verbatim}
            
    \begin{center}
    \adjustimage{max size={0.9\linewidth}{0.9\paperheight}}{output_45_2.png}
    \end{center}
    { \hspace*{\fill} \\}
    
    \hypertarget{trimming-the-image}{%
\subsubsection{Trimming the image}\label{trimming-the-image}}

    We can also slice the image and create a new, smaller one. For instance,
say that we want to produce a new image only with the face of the
dark-coloured Python. To do so, we first need to analyse the image and
calculate the approximate position of the \textbf{top-left} corner of
the square we want to cut. In this case, it could be the coordinate in
the \textbf{row 8} and in the \textbf{column 18}. Then, we need to see
where we want that square to end (i.e. \textbf{bottom-right} corner). In
this case, it is around \textbf{row 24} and \textbf{column 36}. Now, we
can use the following code to produce a trimmed image called
\textbf{img\_trimdark} with the face of the dark-coloured Python.

    \begin{Verbatim}[commandchars=\\\{\}]
{\color{incolor}In [{\color{incolor}18}]:} \PY{n}{img\PYZus{}trimdark} \PY{o}{=} \PY{n}{img}\PY{p}{[}\PY{l+m+mi}{8}\PY{p}{:}\PY{l+m+mi}{24}\PY{p}{,}\PY{l+m+mi}{18}\PY{p}{:}\PY{l+m+mi}{36}\PY{p}{]} \PY{c+c1}{\PYZsh{} rows go first, columns go after}
         \PY{n}{plt}\PY{o}{.}\PY{n}{imshow}\PY{p}{(}\PY{n}{img\PYZus{}trimdark}\PY{p}{)}
\end{Verbatim}


\begin{Verbatim}[commandchars=\\\{\}]
{\color{outcolor}Out[{\color{outcolor}18}]:} <matplotlib.image.AxesImage at 0x15070b0>
\end{Verbatim}
            
    \begin{center}
    \adjustimage{max size={0.9\linewidth}{0.9\paperheight}}{output_48_1.png}
    \end{center}
    { \hspace*{\fill} \\}
    
    Now it's your turn to create a trimmed image called
\textbf{img\_trimlight} with the face of the light-coloured python.

    \begin{Verbatim}[commandchars=\\\{\}]
{\color{incolor}In [{\color{incolor}19}]:} \PY{c+c1}{\PYZsh{}\PYZsh{} Use this cell to create img\PYZus{}trimlight}
         \PY{n}{img\PYZus{}trimlight} \PY{o}{=} \PY{n}{img}\PY{p}{[}\PY{l+m+mi}{55}\PY{p}{:}\PY{l+m+mi}{70}\PY{p}{,}\PY{l+m+mi}{20}\PY{p}{:}\PY{l+m+mi}{56}\PY{p}{]} \PY{c+c1}{\PYZsh{} rows go first, columns go after}
         \PY{n}{plt}\PY{o}{.}\PY{n}{imshow}\PY{p}{(}\PY{n}{img\PYZus{}trimlight}\PY{p}{)}
\end{Verbatim}


\begin{Verbatim}[commandchars=\\\{\}]
{\color{outcolor}Out[{\color{outcolor}19}]:} <matplotlib.image.AxesImage at 0x1543a10>
\end{Verbatim}
            
    \begin{center}
    \adjustimage{max size={0.9\linewidth}{0.9\paperheight}}{output_50_1.png}
    \end{center}
    { \hspace*{\fill} \\}
    
    \hypertarget{inverting-the-image}{%
\subsubsection{Inverting the image}\label{inverting-the-image}}

    We can also create the \emph{negative} of an image by doing the
following:

    \begin{Verbatim}[commandchars=\\\{\}]
{\color{incolor}In [{\color{incolor}20}]:} \PY{n}{img\PYZus{}neg} \PY{o}{=} \PY{l+m+mi}{1} \PY{o}{\PYZhy{}} \PY{n}{img}
         \PY{n}{plt}\PY{o}{.}\PY{n}{imshow}\PY{p}{(}\PY{n}{img\PYZus{}neg}\PY{p}{)}
\end{Verbatim}


\begin{Verbatim}[commandchars=\\\{\}]
{\color{outcolor}Out[{\color{outcolor}20}]:} <matplotlib.image.AxesImage at 0x1588030>
\end{Verbatim}
            
    \begin{center}
    \adjustimage{max size={0.9\linewidth}{0.9\paperheight}}{output_53_1.png}
    \end{center}
    { \hspace*{\fill} \\}
    
    By subtracting 1 minus the original image, basically what we are doing
is converting all ones to zeros, all zeros to ones, and all values in
between to their opposite value in terms of how far they are to 0 or 1.

    \hypertarget{transposing-the-image}{%
\subsubsection{Transposing the image}\label{transposing-the-image}}

    There is a mathematical operation called \textbf{transpose} which lets
you exchange the columns and the rows of a matrix. This function is also
included in the numpy module, and it can be applied to image arrays as
well:

    \begin{Verbatim}[commandchars=\\\{\}]
{\color{incolor}In [{\color{incolor}21}]:} \PY{n}{img\PYZus{}trans} \PY{o}{=} \PY{n}{img}\PY{o}{.}\PY{n}{transpose}\PY{p}{(}\PY{p}{)}
         \PY{n}{plt}\PY{o}{.}\PY{n}{imshow}\PY{p}{(}\PY{n}{img\PYZus{}trans}\PY{p}{)}
\end{Verbatim}


\begin{Verbatim}[commandchars=\\\{\}]
{\color{outcolor}Out[{\color{outcolor}21}]:} <matplotlib.image.AxesImage at 0x15c4270>
\end{Verbatim}
            
    \begin{center}
    \adjustimage{max size={0.9\linewidth}{0.9\paperheight}}{output_57_1.png}
    \end{center}
    { \hspace*{\fill} \\}
    
    Notice that the new image is turned 90 degrees counter-clockwise and has
also been flipped vertically.

    \hypertarget{binarising-the-image}{%
\subsubsection{Binarising the image}\label{binarising-the-image}}

    In computer vision, there are times where you want the image to be just
black and white instead of grayscale. This is useful when you want to
clearly recognise shapes or when you want to get rid of noise. In this
case, image binarisation could be used with two purposes: * Make both
pythons black by converting all gray pixels into black. * Make the light
python disappear by converting all light gray pixels into white.

    Remember you wrote down the values of the pixels of both pythons? Now
it's time to put this information into use! To design our binarisation
function, we need to consider the following: 1. The function has two
inputs: the \textbf{image to binarise} and the \textbf{threshold}, which
is a cut-off value between 0 and 1. 2. The function has to iterate the
numpy array for all rows and columns. 3. For each position, the system
has to compare the value of the pixel with the threshold. If the value
of the pixel is \emph{smaller} than the threshold, then we change the
value of the pixel to zero. Else, we change it to one. 4. To avoid
losing the original image, you can create a copy of it and change the
copy, not the original one! (hint: use the .copy() function to do so).
5. Notice that depending on the threshold value selected, you will get
different outputs!

    \begin{Verbatim}[commandchars=\\\{\}]
{\color{incolor}In [{\color{incolor}22}]:} \PY{c+c1}{\PYZsh{}\PYZsh{} Use this cell to create the binarisation function}
         \PY{k}{def} \PY{n+nf}{binarise}\PY{p}{(}\PY{n}{img}\PY{p}{,} \PY{n}{threshold}\PY{p}{)}\PY{p}{:}
             \PY{l+s+sd}{\PYZsq{}\PYZsq{}\PYZsq{}This function biniarises an image in a numpy array according to the specified threshold\PYZsq{}\PYZsq{}\PYZsq{}}
             \PY{n}{img\PYZus{}bin} \PY{o}{=} \PY{n}{img}\PY{o}{.}\PY{n}{copy}\PY{p}{(}\PY{p}{)}
             \PY{k}{for} \PY{n}{x} \PY{o+ow}{in} \PY{n+nb}{range}\PY{p}{(}\PY{n}{img}\PY{o}{.}\PY{n}{shape}\PY{p}{[}\PY{l+m+mi}{0}\PY{p}{]}\PY{p}{)}\PY{p}{:}
                 \PY{k}{for} \PY{n}{y} \PY{o+ow}{in} \PY{n+nb}{range}\PY{p}{(}\PY{n}{img}\PY{o}{.}\PY{n}{shape}\PY{p}{[}\PY{l+m+mi}{1}\PY{p}{]}\PY{p}{)}\PY{p}{:}
                     \PY{k}{if} \PY{n}{img}\PY{p}{[}\PY{n}{x}\PY{p}{,}\PY{n}{y}\PY{p}{]}\PY{o}{\PYZlt{}}\PY{n}{threshold}\PY{p}{:}
                         \PY{n}{img\PYZus{}bin}\PY{p}{[}\PY{n}{x}\PY{p}{,}\PY{n}{y}\PY{p}{]}\PY{o}{=}\PY{l+m+mi}{0}
                     \PY{k}{else}\PY{p}{:}
                         \PY{n}{img\PYZus{}bin}\PY{p}{[}\PY{n}{x}\PY{p}{,}\PY{n}{y}\PY{p}{]}\PY{o}{=}\PY{l+m+mi}{1}
             \PY{n}{plt}\PY{o}{.}\PY{n}{imshow}\PY{p}{(}\PY{n}{img\PYZus{}bin}\PY{p}{)}
             \PY{k}{return} \PY{n}{img\PYZus{}bin}
\end{Verbatim}


    \begin{Verbatim}[commandchars=\\\{\}]
{\color{incolor}In [{\color{incolor}23}]:} \PY{c+c1}{\PYZsh{}\PYZsh{} Use this cell to execute the binarise function and show both pythons black.}
         \PY{n}{img\PYZus{}bin1} \PY{o}{=} \PY{n}{binarise}\PY{p}{(}\PY{n}{img}\PY{p}{,} \PY{l+m+mf}{0.7}\PY{p}{)}
\end{Verbatim}


    \begin{center}
    \adjustimage{max size={0.9\linewidth}{0.9\paperheight}}{output_63_0.png}
    \end{center}
    { \hspace*{\fill} \\}
    
    \begin{Verbatim}[commandchars=\\\{\}]
{\color{incolor}In [{\color{incolor}24}]:} \PY{c+c1}{\PYZsh{}\PYZsh{} Use this cell to execute the binarise function and show the light python white.}
         \PY{n}{img\PYZus{}bin2} \PY{o}{=} \PY{n}{binarise}\PY{p}{(}\PY{n}{img}\PY{p}{,} \PY{l+m+mf}{0.25}\PY{p}{)}
\end{Verbatim}


    \begin{center}
    \adjustimage{max size={0.9\linewidth}{0.9\paperheight}}{output_64_0.png}
    \end{center}
    { \hspace*{\fill} \\}
    
    \hypertarget{saving-the-image}{%
\subsubsection{Saving the image}\label{saving-the-image}}

    You can save any of the images produced in this tutorial by using the
\textbf{plt.imsave} command. For instance, if we want to save the
transpose image, we can run the following command:

    \begin{Verbatim}[commandchars=\\\{\}]
{\color{incolor}In [{\color{incolor}25}]:} \PY{n}{plt}\PY{o}{.}\PY{n}{imsave}\PY{p}{(}\PY{l+s+s1}{\PYZsq{}}\PY{l+s+s1}{logotrans.png}\PY{l+s+s1}{\PYZsq{}}\PY{p}{,}\PY{n}{img\PYZus{}trans}\PY{p}{)}
\end{Verbatim}


    You will notice that the image get saved with the yellow background! Is
there any other way to save them?

    \begin{Verbatim}[commandchars=\\\{\}]
{\color{incolor}In [{\color{incolor}26}]:} \PY{c+c1}{\PYZsh{}\PYZsh{} Use this cell to look for other ways to save the images with better quality.}
         \PY{k+kn}{import} \PY{n+nn}{imageio}
         \PY{n}{imageio}\PY{o}{.}\PY{n}{imwrite}\PY{p}{(}\PY{l+s+s1}{\PYZsq{}}\PY{l+s+s1}{logotranshd.png}\PY{l+s+s1}{\PYZsq{}}\PY{p}{,} \PY{n}{img\PYZus{}trans}\PY{p}{)}
\end{Verbatim}


    \begin{Verbatim}[commandchars=\\\{\}]
D:\textbackslash{}Anaconda3\textbackslash{}lib\textbackslash{}site-packages\textbackslash{}imageio\textbackslash{}core\textbackslash{}util.py:78: UserWarning: Lossy conversion from float64 to uint8, range [0, 1]
  dtype\_str, out\_type.\_\_name\_\_))

    \end{Verbatim}


    % Add a bibliography block to the postdoc
    
    
    
    \end{document}
